This report is a short description of my six month internship carried out as a
compulsory component of the BSc. Electronic and Electrical Engineering. The
internship was carried out at Grit Systems Engineering, Victoria Island, Lagos.
This internship report contains my activities that have contributed to achieving a
successful program. In this chapter, a description of Students Industrial Work Experience
Scheme (SIWES) and the Company is given. The second chapter contains the
theoretical background of the activities carried out during the course of the training,
followed by the third chapter which discusses the activities. The technical experience
and skills acquired during the internship are described in the fourth chapter. Finally I
give a conclusion on the internship experience.

\section[SIWES]{Students Industrial Work Experience Scheme (\textbf{SIWES})}

The SIWES is a program designed for students in their third and fourth year from the faculty of Technology and
Environmental Design and Management. The program aims at inculcating practical,
scientific, social, and entrepreneurship skills needed to face the challenges of modern
day graduate while also contributing to the overall development of undergraduates in
these faculties. SIWES is highly recognized by the Nigerian University Commission
(NUC), National Board for Technical Education (NBTE) and National Commission
for Colleges of Education (NCCE), which makes her join forces with Industrial
Training Fund (ITF) in making this program attainable.

\section{Objective of SIWES} 

The goals that the SIWES seek to accomplish are as follows
\begin{itemize}
\item Bridge the gap between the theoretical work/knowledge acquired in the
classroom and real practical experience offered in the industries. This enables
the students to appreciate/value in so many ways, the theories learnt in class.
\item To enlist and strengthens employers involvement in the entire educational
process of preparing graduate for employment in the industry.
\item To provide students with an opportunity to apply their theoretical knowledge
in real work situation, thereby bridging the gap between university work and
actual practice.
\item To prepare the student for the challenges in the industries and prepare them
psychologically for work after school.
\end{itemize}

\section{GRIT Systems}
GRIT Systems, formally DawnFuel Limited, was founded by Mr. Ifedayo
Oladapo in 2011 after he identified the challenge of under electrification in Nigeria. He
demonstrated that solar power costs less than running a generator does, demonstrated
it and used it as economic reality and use it as the foundation for the case he made for
DawnFuel. As well as in the company’s marketing campaign.
The company after a while settled for installation and maintenance of imported solar
products, with its custom built components, as the market did not appreciate locally
built inverters. \\
\\
GRIT Systems got into consultancy and monitoring of energy utilization with its new line of products.These products include The GRIT Energy and Power Monitor (GEPM) and G1 (A utility metering device). The company started with just 2 members (including the founder) and grew into a multi-departmental company with more than 15 workers. The devices were tailored to the unique requirements of under electrified communities and
requires a first-hand installation by a trained Grit System’s personnel. Once installed, users can remotely view graphs, receive notifications and generate simple language reports about an arbitrarily complex power supply mix.\\
Some of the functions of the Grit meter are:
\begin{itemize}
\item Reduced energy cost - Ensuring generator only runs when it is really needed.
\item Multisource energy optimization – Increase in the time spent on cost effective
sources like inverter while reducing the time spent on expensive sources like
your generator.
\item Cost-Benefit Balance - Using data from the metering devices to run energy balance
simulations to help determine if and how alternative power sources would
save the user money. 
\end{itemize}
